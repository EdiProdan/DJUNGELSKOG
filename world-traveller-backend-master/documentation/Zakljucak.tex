\chapter{Zaključak i budući rad}
		
		%\textbf{\textit{dio 2. revizije}}\\
		
		%\textit{U ovom poglavlju potrebno je napisati osvrt na vrijeme izrade projektnog zadatka, koji su tehnički izazovi prepoznati, jesu li riješeni ili kako bi mogli biti riješeni, koja su znanja stečena pri izradi projekta, koja bi znanja bila posebno potrebna za brže i kvalitetnije ostvarenje projekta i koje bi bile perspektive za nastavak rada u projektnoj grupi.}
		
		 %\textit{Potrebno je točno popisati funkcionalnosti koje nisu implementirane u ostvarenoj aplikaciji.}

            Zadatak grupe "DJUNGELSKOG" je bio razvoj web aplikacije za praćenje putovanja po svijetu i dijeljenje istih sa prijateljima. Nakon 17 tjedana rada na aplikaciji ostvaren je cilj i aplikacija "Svjetski putnik" je u potpunosti izrađena.
            \\
            Izradu projekta možemo promatrati u dvije faze, svaka odgovara jednom ciklusu predavanja semestra. U prvoj fazi primaran cilj je bio upoznat se i osmisliti kako ostvariti aplikaciju te postići funkcionalnu alfa inačicu web aplikacije. 
            Česti sastanci tima su omogućili dobru organiziranost i koliko god moguću ravnomjernu raspodijelu posla unutar tima.
            \\
            Druga faza projekta je bila različita od prve, primarno je bilo jako puno intenzivnog rada kako bi se ispunili \textit{deadline}-ovi. Dio tima se prvi put susreo sa određenim tehnologijama te je bilo potrebno uložiti dosta vremena kako bi ih se naučilo koristiti. S obzirom na dobar plan iz prve faze projekta, u drugoj fazi je tim samo trebao pratiti navedeni plan što je u konačnici dovelo do gotovog proizvoda.
            \\
            Kroz obje faze su se izrađivali UML dijagrami koji opisuju funkcionalnost aplikacije iz različitih perspektiva. Osim dijagrama izrađena je dokumentacija koja opisuje velik dio funkcionalnosti projekta i služi budućim korisnicima kao pomoć pri uporabi aplikacije ili izmjenama aplikacije.
            \\
            Moguće proširenje postojeće inačice sustava je izrada mobilne aplikacije te dodavanje 3D modela posjećenih objekata bedževima kako bi se postigla zanimljivija estetika.
            \\
            Sudjelovanje na ovakvom projektu omogućilo je svim članovima tima stjecanje vrijednog iskustva rada u timu te upoznavanja novih tehnologija i dokumentiranja većeg projekta. Kao tim zadovoljni smo postignutim rješenjem i uspjehom jer smo uspjeli ostvariti cilj uz sve druge obaveze koje svaki član tima ima.
            \\
            
		
		\eject  