\chapter{Opis projektnog zadatka}
		
		Cilj ovog projekta je razviti programsku podršku za stvaranje aplikacije "\textit{Svjetski putnik"} koja se ponaša kao društvena mreža na kojoj korisnik osvaja bedževe svojim putovanjima po svijetu, osvajanje što više teritorija na karti te što više bedževa je glavni cilj korisniku.
		
		Primjer osvajanja bedža:
		\begin{packed_item}
		    \item{Bedž države - \textit{posjeti glavni grad i dvije lokacije van glavnog grada}}
		    \item{Bedž grada - \textit{posjeti minimalno 5 lokacija unutar grada}}
		\end{packed_item}
		
		Pokretanjem aplikacije prikazuje se ekran za prijavu, odnosno registraciju korisnika. Za kreiranje novog računa (registraciju korisnika) potrebni su podatci:
		\begin{packed_item}
			\item{korisničko ime}
			\item{ime korisnika}
			\item{prezime korisnika}
			\item{email adresa}
			\item{lozinka}
		\end{packed_item}
		
		Ovakvom registracijom korisnik dobiva prava regularnog korisnika no naknadno mu od strane administratora mogu biti dodijeljene korisničke uloge administratora i/ili kartografa. Prijavom u aplikaciju se dolazi na početnu stranicu (engl. Home page).
		
		Postoje 3 tipa korisničkih uloga:
		\begin{packed_item}
			\item{regularni korisnik}
			\item{administrator}
			\item{kartograf}
		\end{packed_item}
		
		\section{Korisničke uloge}
		\textbf{\textit{Regularni korisnik}} ima pregled svojih prošlih putovanja na karti tako da su države koje je obišao označene pinom te obojene a one koje još nije su sive boje. Korisnik može dodavati putovanja na wishlistu uz definiranje roka do kada mora obići mjesto. Regularni korisnik može pregledavati profil i osvojene bedževe drugih korisnika, te definirati prikazuje li javno svoja putovanja ili samo bedževe koje je osvojio. 
		
		
		Obilaskom lokacije korisnik treba unijeti kao dokaz sliku sebe na toj lokaciji, te ispuniti formu u kojoj se bilježi:
		\begin{packed_item}
		    \item{datum posjeta}
		    \item{vrsta prijevoza do mjesta}
		    \item{ocjena gužve}
		    \item{samostalno putovanje ili u društvu}
		    \item{recenzija putovanja od 1 do 5}
		    \item{komentar na putovanje}
		\end{packed_item}
		
		\textbf{\textit{Kartograf}} ima pravo definiranja pravila za osvajanje bedževa, ali ne i ovlast za uređivanje ili uklanjanje postojećih.
		
		\textbf{\textit{Administrator}} ima pravo unositi države, gradove i lokacije unutar grada (objekti poput muzeja, crkvi...), odobravati prijedloge korisnika za unos novih lokacija te brisanje korisnika iz sustava.
		
		\section{Korist, opseg te mogući skup korisnika projekta}
		\subsection{Korist projekta}
		Korist ovog projekta bi bila motivirati ljude zanimljivim bedževima da putuju više i posjete nove lokacije za koje možda nisu ni znali da postoje a saznali su preko bedževa, također promovira povezanost ljudi tako da omogućuje pregled i reagiranje na putovanja prijatelja.
		Skup mogućih korisnika su mladi ljudi koji često putuju i žele dijeliti svoja putovanja sa svojim prijateljima
		
		\subsection{Opseg projekta}
		Projekt obuhvaća ostvarenje karte po kojoj je moguće navigirati, stavljati pinove i predlagati nove lokacije, te ostvarenje društvenog aspekta aplikacije nalik društvene mreže u kojem je moguće dodavati prijatelje, pretraživati korisnike i reagirati na sadržaj (putovanja) prijatelja. U aplikaciji se osvajaju bedževi za ispunjavanje određenih zahtjeva (pravila) svakog bedža koje kreira kartograf. Osim bedževa koje je kartograf izradio, korisnik samostalno može dodavati određene lokacije na wishlist kako bi ih kasnije posjetio.
		
		
		\eject
		
	